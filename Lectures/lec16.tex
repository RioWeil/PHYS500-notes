\section{Midterm Review}

\subsection{Two Spin-1 Particles}
We consider a system made of two spin-1 particles with the Hamiltonian:
\begin{equation}
    H = \e_1 \v{S}^{(1)} \cdot \v{S} + \e_2(S_z)^2
\end{equation}
where $\v{S} = \v{S}^{(1)} + \v{S}^{(2)}$, $\e_1 > 0, \e_2 > 0$. 

\begin{enumerate}[(a)]
    \item Demonstrate (by computing the relevant commutators) that the classification $\ket{S^{(1)}, S^{(2)}, S, S_z}$ is a good classification scheme of states for this Hamiltonian.
    \item What is the lowest energy of the system?
    \item What is the ground state? Formulate your answer using the classification $\ket{S^{(1)}, S^{(2)}, S, S_z}$. Use CG coefficients to represent this state in terms of $\ket{S^{(1)}, S^{(2)}, S_z^{(1)}, S_z^{(2)}}$. 
    \item Compute the expectation value $\avg{S_x^{(1)}S_x^{(2)} + S_y^{(1)}S_y^{(2)}}$ for the ground state.
    \item \textbf{Strictly qualitative problem. No calculations!} What is the expectation value $\avg{S_z^{(1)}}$ for the ground state? Zero or nonzero? Present the arguments.
\end{enumerate}

\noindent
\textit{Solution.} 
\begin{enumerate}[(a)]
    \item We start by rewriting the Hamiltonian:
    \begin{equation}
        H = \e_1 (\v{S}^{(1)})^2 + \e_1\v{S}^{(1)} \cdot \v{S}^{(2)} + \e_2 (S_z)^2
    \end{equation}
    We can easily find that $[H, S^2] = 0$ and $[H, S_z] = 0$, $[H, (S^{(1)})^2] = [H, (S^{(2)})^2] = 0$ by writing $\v{S}^{(1)} \cdot \v{S}^{(2)} = \frac{1}{2}(S^2 - (\v{S}^{(1)})^2 - (\v{S}^{(2)})^2)$. Note that $[H, S_i] = 0$. 
    \item We rewrite:
    \begin{equation}
        H = \frac{\e_1}{2}\left(S_1^2 - S_2^2\right) + \frac{\e_1}{2}S^2 + \e_2 S_z^2
    \end{equation}
    $S_1^2 - S_2^2$ is invariant, so we pick a state such that $S^2, S_z^2$ are minimized. So we pick the state with $s = 0$ and $s_z = 0$. but then by symmetry $S_1^2 - S_2^2$ will be zero. So the ground state energy is zero.
    \item In the eigenbasis we have $\ket{s = 0, s_z = 0}$. We can use a CG table to write:
    \begin{equation}
        \ket{s = 0, s_z = 0} = \frac{1}{\sqrt{3}}\ket{s_z^{(1)} = -1, s_z^{(2)} = 1} + \frac{1}{\sqrt{3}}\ket{s_z^{(1)} = 0, s_z^{(2)} = 0} + \frac{1}{\sqrt{3}}\ket{s_z^{(1)} = 1, s_z^{(2)} = -1}.
    \end{equation}
    \item The easiest way to do this is to write:
    \begin{equation}
        \avg{S_x^{(1)}S_x^{(2)} + S_y^{(1)}S_y^{(2)}} = \avg{\v{S}^{(1)} \cdot \v{S}^{(2)} - S_z^{(1)}S_z^{(2)}}
    \end{equation}
    We then have:
    \begin{equation}
        \v{S}^{(1)} \cdot \v{S}^{(2)} = \frac{1}{2}(S^2 - (S^{(1)})^2 - (S^{(2)})^2)
    \end{equation}
    so computing the values using the relevant eigenbases:
    \begin{equation}
        \avg{S_x^{(1)}S_x^{(2)} + S_y^{(1)}S_y^{(2)}} = \frac{1}{2}(0 - 2 - 2) + \frac{2}{3} = -\frac{4}{3}
    \end{equation}
    \item $\avg{S_z^{(1)}} = 0$. Because a singlet is rotationally invariant, there is no preferred direction for a single spin.
\end{enumerate}


\subsection{Perturbation Theory for Two-Level System}
Cosnider a two-state system with a Hamiltonian represented in the spin $S_z$ basis as
\begin{equation}
    H^{(0)} = \m{a & 0 \\ 0 & b}.
\end{equation}
Now the system is perturbed with
\begin{equation}
    H' = \m{a_3 & a_1 \\ a_1 & -a_3}
\end{equation}
where $a_1 \approx a_3 \ll a, b$. 
\begin{enumerate}[(a)]
    \item Show that the unperturbed system described two levels with energies $E_1 = a$ and $E_2 = b$ correspondingly.
    \item Using first order perturbation theory to compute corrections to these energy levels.
    \item Compute the exact eigenenergies by diagonalizing the full Hamiltonian $H^{(0)} + H'$. 
    \item Use Taylor expansion from (c) to reproduce the first order correction from item (b).
    \item What is the accuracy of the perturbation theory? In other words what are the (leading) corrections which had been neglected in computations (b)?
    \item Take $a = b$ in exact formula derived in (c). Do you reproduce first order perturbation result derived in (b)? Make comments.
\end{enumerate}

\noindent
\textit{Solution.}

\begin{enumerate}[(a)]
    \item Just read this off. $H^{(0)}$ is a diagonal matrix with entries (and hence) eigenvalues $a, b$. 
    \item Use first order perturbation theory to compute corrections to these energy levels. Assuming $a \neq b$, we can use non-degenerate PT to compute:
    \begin{equation}
        \Delta E_1 = \m{1 & 0}\m{a_3 & a_1 \\ a_1 & -a_3}\m{1\\ 0} = a_3
    \end{equation}
    \begin{equation}
        \Delta E_2 = \m{0 & 1}\m{a_3 & a_1 \\ a_1 & -a_3}\m{0\\ 1} = -a_3
    \end{equation}
    \item Adding:
    \begin{equation}
        H = H^{(0)} + H' = \m{a + a_3 & a_1 \\ a_1 & b - a_3}
    \end{equation}
    then:
    \begin{equation}
        \begin{split}
            \det(H - E_{1,2}\mathbb{I}) = 0 \implies E_{1, 2} &= \frac{a+b}{2} \pm \sqrt{\left(\frac{a+b}{2}\right)^2 - ((a + a_3)(b - a_3) - a_1^2)}
            \\ &= \frac{a + b}{2} \pm \sqrt{\left(\frac{a+b}{2}\right)^2 - ab - a_3(b - a) + a_1^2 + a_3^2}
            \\ &= \frac{a+b}{2} \pm \sqrt{\left(\frac{a-b}{2}\right)^2 + a_3(a - b) + a_1^2 + a_3^2}
            \\ &= \frac{a+b}{2} \pm \frac{a-b}{2}\sqrt{1 + \frac{4a_3}{a-b} + \frac{4(a_1^2 + a_3^2)}{a-b}}
        \end{split}
    \end{equation}
    \item Taylor expanding (to first order)
    \begin{equation}
        E_{1, 2} \approx \frac{a + b}{2} \pm \frac{a-b}{2}\left(1 + \frac{2a_3}{a-b}\right) = \frac{a + b}{2} \pm \frac{a - b}{2} \pm a_3 \implies E_1 \approx a + a_3, E_2 \approx b - a_3
    \end{equation}
    \item The corrections are of the order:
    \begin{equation}
        \mathcal{O}(\frac{a_1^2/a_3^2}{a-b})
    \end{equation}
    so long as $a_1 \ll a - b$.
    \item When we take $a = b$, we have:
    \begin{equation}
        H = \m{a + a_3 & a_1 \\ a_1 & a - a_3} \implies \lambda_{1, 2} = a \pm \sqrt{a_1^2 + a_3^2}
    \end{equation}
    and we don't get the same corrections, because we need to use degenerate PT instead of non-degenerate PT here, of course.
\end{enumerate}
