\section{WKB Part III}

\subsection{Review of the Connection Formula Derivation}
Let us explain what we have derived. Let assume a setup as in Fig. \ref{fig-connectionformula}. The wavefunction in the neighbourhood around the turning point $U = E$ (let us call this point $x_2$) cannot be described by WKB. But if $\psi_2$ (in the calssically forbidden region) is described by the form:
\begin{equation}
    \psi_2(x) = \frac{D}{\sqrt{p}}e^{-\frac{1}{\hbar}\int_{x_2}^x p(x')dx'}
\end{equation}
then $\psi_1(x)$ in the classically allowed region has the structure:
\begin{equation}
    \psi_1(x) = \frac{2D}{\sqrt{p}}\sin\left(\frac{1}{\hbar}\int_{x}^{x_2} p(x')dx' + \frac{\pi}{4}\right)
\end{equation}

Let us make a simple point; the fact that a sine appears in the above is absolutely trivial; we have imaginary exponential solutions to the SE in the classically allowed region. The nontrivial part is the $+\frac{\pi}{4}$ phase appears.

\subsection{Deriving the Bohr-Sommerfield Quantization Formula}
We can repeat exactly the same arguments for a setup like Fig. \ref{fig-connectionformula} but with the classically forbidden region to the left and the classically allowed region to the right of a turning point $x_1$. We then have that $\psi(x)$ to the left of the turning point is:
\begin{equation}
    \psi_{x < x_1}(x) = \frac{D'}{\sqrt{p}}e^{-\frac{1}{\hbar}\int_x^{x_1} p(x')dx'}
\end{equation}
note the order of the integral which ensures the decay of the wavefunction at infinity. We get an extremely similar formula for the wavefunction in the classically allowed region:
\begin{equation}
    \psi_{x > x_1}(x) = \frac{2D'}{\sqrt{p}}\sin\left(\frac{1}{\hbar}\int_{x_1}^{x} p(x')dx' + \frac{\pi}{4}\right)
\end{equation}

We now present a non-trivial claim: We claim that we can use the connection formulas going from both sides (assume we have $U$ which such that $x_1 < x < x_2$ is a classically allowed region inside $x_1 > x$ and $x > x_2$ are classically forbidden). When we do this, the wavefunctions inside of the classically allowed region \emph{must agree}. We therefore obtain the condition:
\begin{equation}
    \sin\left(\frac{1}{\hbar}\int_{x_1}^{x} p(x')dx' + \frac{\pi}{4}\right) = \pm \sin\left(\frac{1}{\hbar}\int_{x}^{x_2} p(x')dx' + \frac{\pi}{4}\right)
\end{equation}
where the $\pm$ comes from $D' = \pm D$. We now consider writing the integral on the LHS as:
\begin{equation}
    \int_{x_1}^x + \frac{\pi}{4} = \int_{x_2}^x  + \int_{x_1}^{x_2} + \frac{\pi}{4} = -\int_{x}^{x_2} + \int_{x_1}^{x_2} + \frac{\pi}{4} - \frac{\pi}{4} + \frac{\pi}{4}
\end{equation}
We then see that $-\int_{x}^{x_2} - \frac{\pi}{4}$ precisely appears on the RHS, and so we obtain the \emph{Bohr-Sommerfield quantization condition}:
\begin{equation}
    \theta = \int_{x_1}^{x_2} + \frac{\pi}{2} = n\pi
\end{equation}
or rewriting this:
\begin{equation}
    \boxed{\int_{x_1}^{x_2}\frac{p(x')dx'}{\hbar} = \pi(n + \frac{1}{2})}
\end{equation}

\subsection{Analyzing the Bohr-Sommerfield Quantization Formula}
\begin{enumerate}[(a)]
    \item What is $n$? This is the number of nodes inside the potential, or also the excitation number.
    \item We can reformulate the condition as:
    \begin{equation}
        n = \int_{x_1}^{x_2}\frac{pdx}{\hbar \pi} = \oiint_{\frac{P^2}{2m} \leq E - U} \frac{dpdx}{2\pi\hbar} = \oint \frac{pdx}{2\pi\hbar}
    \end{equation}
    So what does this say? We take the phase volume of the system, and divide this by $2\pi\hbar$; We therefore obtain the nuber of quantum cells! 
    \item We find that one quantum cell has volume $2\pi\hbar$ (this could also follow by considering the uncertainty relation).
    \item In $3D$, we can generalize everything we have just said, and write:
    \begin{equation}
        dn = \frac{d^3pd^3x}{(2\pi\hbar)^3} = \frac{V}{(2\pi\hbar)^3}p^2dpd\Omega
    \end{equation}
    Often we can represent $p$ in terms of energy, and find the number of states per unit energy (density of states in terms of energy). Explicitly, we can see that $p^2 dp = pEdp$ which follows from:
    \begin{equation}
        E^2 = p^2  +m^2 \implies EdE = pdp
    \end{equation}
    This is correct for both nonrelativistic and relativistic physics (we will use this formula to study photons). For free particles with $E = p^2/2m$, we find:
    \begin{equation}
        \frac{1}{V}\frac{dn}{dE} = \frac{d\Omega}{(2\pi\hbar)^3}\abs{p}E
    \end{equation}
    which is the number of states per unit energy per unit volume. Density of states is extremely useful, and comes up in many places in physics, such as:
    \begin{itemize}
        \item Fermi liquid model in condensed matter physics
        \item Transitions in atomic physics
        \item Thomas-Fermi models for complex models.
    \end{itemize}
\end{enumerate}

\subsection{Solving the QHO with WKB}
Let us use WKB to solve the simplest problem there is; the Harmonic Oscillator. We want to quantize it assuming we don't know the exact solution. Writing down the Bohr-Sommerfield quantization condition, we find:
\begin{equation}
    \int_{-a}^a pdx = \pi\hbar(n + \frac{1}{2})
\end{equation}
We have $p = \sqrt{2m(E - U(x))}$ and $U(x) = \frac{1}{2}kx^2$. At $x = \pm a$ we have $E = \frac{ka^2}{2}$ and so:
\begin{equation}
    \int_{-a}^{a} pdx = a\sqrt{2mE}\int_{-a}^{a}\sqrt{1 - \left(\frac{x}{a}\right)^2}\frac{dx}{a}
\end{equation}
Now making the substitution $z = \frac{x}{a}$ with $z \in (-1, 1)$, we can evaluate the integral (e.g. via trigonometric substitution) to find:
\begin{equation}
    \int_{-a}^{a} pdx  = \sqrt{\frac{2E}{k}}\sqrt{2mE}\frac{\pi}{2} = \pi\hbar(n + \frac{1}{2})
\end{equation}
With $\omega^2 = k/m$, we find:
\begin{equation}
    E = \hbar\omega(n + \frac{1}{2})
\end{equation}
and we reproduce exactly the analytical result for the eigenenergies of the oscillator! Nevertheless, we cannot trust this result for sufficiently small $n$. How do we find out when this formula can be trusted? We go back to the discussion of when WKB is justified:
\begin{equation}
    \left|\dod{}{x}\frac{\hbar}{p}\right| \ll 1
\end{equation}
Differentiating, we find:
\begin{equation}
    \left|\frac{\hbar}{\sqrt{2m}}\frac{1}{(E - U)^{3/2}}\frac{1}{2}kx\right| \ll 1
\end{equation}
which with $U = \frac{1}{2}kx^2$, and $z = x/a$ becomes;
\begin{equation}
    \left| \frac{\hbar\omega}{E}\frac{z}{(1-z^2)^{3/2}}\right| \ll 1
\end{equation}
substituting in $E_n = \hbar\omega(n + \frac{1}{2})$ (and ignoring some coefficients) we have:
\begin{equation}
    \left|\frac{1}{(1-z^2)^{3/2}}\frac{1}{n}\right| \ll 1
\end{equation}
so for sufficiently large $n$, so long as we stay away from the turning points $z = \pm 1$ (i.e. $x = \pm a$) WKB is reliable here. How close we can get to the turning point will be a matter of how large $n$ is.

After the midterm on wednesday, we will use WKB to analyze tunnelling.