\section{Review}
\subsection{Resonance Scattering}
We derived a formula for a very simple attractive potential:
\begin{equation}
    U(r) = \begin{cases}
        -U_0 & r \leq a_0
        \\ 0 & r > a_0
    \end{cases}
\end{equation}
we discussed two different cases. We looked at when perturbation theory is justified - It is justified in the limit where $U_0$ is small. We obtained the cross section $\sigma = 4\pi a^2 \left(\frac{1}{3}\chi^2 a^2\right)^2 \ll 4\pi a^2$. In PT, we always get something smaller than the typical size of the system.

We also discussed partial wave analysis. This is a highly generic technique. Wehn $\chi a \neq \pi/2$, then in the $U_0 \to \infty$ limit we have $\sigma = 4\pi a^2$. The intuitition is that for low energy scattering from a strong potential that the particle can scatter off of the entire sphere, so the cross section is just the surface area.

We looked at the case of resonances. What we computed was that the amplitude $f = \frac{1}{\chi \cot \chi a - ik}$. There is the so called scattering length:
\begin{equation}
    \lim_{k \to 0}f = a_0
\end{equation}
We tried to understand how the cross section could be much larger than the actual size of the system. When we have resonance, we have $\sigma = \frac{4\pi}{k^2}$, i.e. the saturation of the unitarity condition. For small $k$, the cross section can be unbelievably large. Where does this come from? We solved for the bound states of the finite square well. When we solve for the bound state problem, we see that our wavefunction decays in a very, very slow fashion, as $e^{-r/a_0}$ where $a_0$ is precisely the scattering length. Generally the particle will be found outside of the potential.

\subsection{Partial Wave Analysis}
We have the partial wave amplitudes:
\begin{equation}
    f_l = \frac{S_l - 1}{2ik} = \frac{e^{i\delta_l}}{k}\sin(\delta_l)
\end{equation}
so then:
\begin{equation}
    \abs{f_l}^2 = \frac{\sin^2\delta_l}{k^2}
\end{equation}
\begin{equation}
    \text{Im}f_l = \frac{\sin^2\delta_l}{k}
\end{equation}
so:
\begin{equation}
    \text{Im}\frac{1}{f_l} = \text{Im}\frac{f_l^*}{\abs{f_l}^2} = \frac{1}{\abs{f_l}^2}\text{Im}f_l^* = -k
\end{equation}
so:
\begin{equation}
    \frac{1}{f_l} = g_l(k) - ik \implies f_l = \frac{1}{g_l - ik}
\end{equation}
And this elucidates the (complex) mathematical structure of the partial wave amplitudes.

\subsection{Transitions}
Recall the transition rate:
\begin{equation}
    \dod{W}{t} = \frac{2\pi}{\hbar}\abs{V_{ab}}^2\delta(E_b - E_a - \hbar \omega)\frac{d^3pd^3x}{(2\pi\hbar)^3}
\end{equation}
we obtained this by perturbation theory - it is valid when the transition rate is small. Fortunately for our life, PT is justified for maxwell's systems, because $\alpha = \frac{1}{137}$.

Then we had for the case of emission:
\begin{equation}
    \dod{W}{t} = \sum_\lambda \frac{2\pi}{\hbar}\left(\frac{e}{mc}\right)^2 \left(\frac{2\pi \hbar c^2}{\omega V}\right)\left(\frac{\omega^2 d\Omega V}{(2\pi)^3 c^3 \hbar}\right)\abs{\bra{f}e^{-i\v{k} \cdot \v{r}}\gv{e}^{\lambda}(\v{k}) \cdot\v{p} \ket{i}}^2
\end{equation}
Note if $ka \sim \frac{a}{\lambda} \sim \alpha \ll 1$, then $e^{-i\v{k} \cdot \v{r}} \sim 1$. Note that for heavy atoms, we have $Z\alpha$ so it is not justified.

We discussed the selection rules, i.e. when (for electric dipole transitions) is $\bra{f}\v{r} \ket{i} = 0$. To this end, we expressed $\v{p} \sim [\v{r}, H]$ (which easily follows from $H = \frac{\v{p}^2}{2m} + V(r)$). We did all of these complex computations, and at the very end we have:
\begin{equation}
    \dod{W}{t} = \frac{4}{3}\frac{\omega^3e^2}{\hbar c^3}\abs{\bra{f}\v{r}\ket{i}}^2
\end{equation}
this agrees with the classical E\&M calculation where one considers a dipole moment (oscillating dipole), then calculate the electric and magnetic field far away, from this the poynting vector $\v{S} = \v{E} \times \v{B}$ can be calculated and integrated over the sphere. Note however to have the quantum classical correspondence we have $I = \hbar \omega \dod{W}{t} \sim \omega^4$ and this is precisely the dependence found in classical physics, $I \sim \ddot{\v{d}}^2 \sim \omega^4$. 

\subsection{Gauge transformations}
We consider:
\begin{equation}
    H = \frac{(\v{p} - \frac{e}{c}\v{A})^2}{2m}
\end{equation}
where $\v{A} = \nabla \Lambda$ is the Gauge, then $\psi = e^{i\frac{e}{\hbar c}\Lambda}\psi$. If we change the gauge, then we introduce a phase and the single-valuedness of the wavefunction is violated - one does not have a single ground states but infinitely many $(2\pi n, n \in \ZZ)$. Each time we make a circle, the complete classification is no longer given just by the phase, you need to remember the number of times you wrapped around.