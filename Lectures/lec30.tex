\section{Resonance Scattering}
\subsection{Review of Prior Results}
This is the last lecture for content - we finish our discussion of scattering by discussing resonance. 

We have our formula for the total cross-section:
\begin{equation}
    \sigma = \frac{4\pi}{k^2}\sum_l (2l + a)\sin^2(\delta_l(k))
\end{equation}

Separately, there is a $\sigma_l$ which is the partial cross section:
\begin{equation}
    \sigma_l = \frac{4\pi}{k^2}\sin^2(\delta_l) \leq \frac{4\pi}{k^2}
\end{equation}
which is always bounded. This is known as the unitarity limit. If we measure something larger than this, there is either something fundamentally wrong with QM, or with the measurement. Sometimes people put the degeneracy $(2l+1)$ in the formula:
\begin{equation}
    \sigma_l = \frac{4\pi}{k^2}(2l+1)\sin^2(\delta_l) \leq (2l+1)\frac{4\pi}{k^2}
\end{equation}
so, we cannot have a cross section for a single DOF larger than $\frac{4\pi^2}{k^2}$.

We also derived the partial wave amplitude to have the extremely generic structure:
\begin{equation}
    f_l = \frac{1}{g_l(k) - ik}
\end{equation}
with $g_l$ a real function. We will see the physical meaning of it very shortly.

In Lecture 28, we solved the problem of scattering from an attractive potential, $U(r) = -U_0$ for $r \leq a$ (and $U(r) = 0$ for $r > a$). Let us study what happens here. We derived the expression for the cross section, which is:
\begin{equation}
    \sigma = 4\pi a^2
\end{equation}
for $U_0 \to \infty$. This is a special case, where $\tan \chi a \neq \infty$ (we ignore the singularities). We will see what happens at the special points in a second. Note this is in stark contrast to what happens in classical physics, where $\sigma = \pi a^2$. We have $4\pi a^2$ because in QM the particle for sufficiently large wavelengths can scatter off of the entire sphere. We have here the amplitude:
\begin{equation}
    f_0 = \frac{1}{k\cot\delta - ik} = \frac{1}{\chi \cot \chi a - ik}
\end{equation}
So, we identify $g_l$ with $\chi \cot \chi a$, and we showed consistency with the generic structure.

We consider the scattering length:
\begin{equation}
    f(k \to 0) = -a_0
\end{equation}
where in our case:
\begin{equation}
    f(k \to 0) = \frac{1}{\chi \cot \chi a}
\end{equation}
Let us rewrite the cross section as:
\begin{equation}
    \sigma = \frac{4\pi a^2}{\left((ka)^2 + (\chi a\cot \chi a)^2\right)}
\end{equation}

Now, let's solve a completely different problem, and try to understand the physical meaning of the scattering length/$g_l$. 

\subsection{Square Well Bound States and Resonance}
We consider exactly the same attractive potential as previous:
\begin{equation}
    U(r) = \begin{cases}
        -U_0 & r \leq a
        \\ 0 & r > a
    \end{cases}
\end{equation}
except now we want to solve for the bound state energies $E < 0$. You definitely did this in PHYS 304 already, but let us try it again. Let us define $\tilde{k}^2 = \frac{2m\abs{E}}{\hbar^2}$ (do not confuse it with $k^2 = \frac{2mE}{\hbar^2}$ which was for strictly positive energies). 

We introduce as before $\chi^2 = \frac{2m\abs{U_0} - 2m\abs{E}}{\hbar^2} = \frac{2m\abs{U_0}}{\hbar^2} - \tilde{k}^2$. 

So we have the same equations as before, but with a small difference:
\begin{equation}
    \begin{cases}
        u'' - \tilde{k}^2 u = 0, \quad r \geq a
        \\ u'' + \chi^2 u = 0, \quad r \leq a
    \end{cases}
\end{equation}
note the minus sign because we have a bound state! the solution will be very different in this case. This has solution:
\begin{equation}
    \begin{cases}
        u(r) = A\sin(kr) \quad r \leq a
        \\ u(r) = B e^{-\tilde{k}r}\quad r \geq a
    \end{cases}
\end{equation}
so from the continuity condition:
\begin{equation}
    \left. \frac{u'}{u}\right|_{r = a + \e} = \left. \frac{u'}{u}\right|_{r = a - \e}
\end{equation}
So:
\begin{equation}
    \chi a \cot \chi a = -\tilde{k}a
\end{equation}
and so:
\begin{equation}
    f = \frac{1}{\chi \cot \chi a - i k} = \frac{1}{-\tilde{k} - ik}
\end{equation}
And the claim is the following:
\begin{equation}
    \chi \cot \chi a = -\frac{1}{a_0}
\end{equation}
Now, we understand what happens for a huge cross section. We have:
\begin{equation}
    f = \frac{1}{-\frac{1}{a_0} - ik}
\end{equation}
And we note:
\begin{equation}
    f(k \to 0) \sim e^{-r/a_0}
\end{equation}
so when $\chi a= \frac{\pi}{2}$, $a_0$ is huge. So, since $e^{-r/a_0}$, then scattering happens on distances much larger than the scale of the potential. This is why $a_0$ is the scattering length. Even though we still have a bound state, when we do scattering, the particles most of the time stay outside the potential, and in fact most of the scattering happens faraway from the potential!

When we have the resonance condition $\chi a = \frac{\pi}{2}$, then $f = \frac{1}{-ik}$ and so $\sigma = \frac{4\pi}{k^2}$ i.e. the unitarity condition is saturated. 

If we look precisely what happens with bound states here, the bound state is extremely close to the $0$ point (i.e. to the top of the potential well). So, we have derived the physical meaning of the resonance.


\subsection{Complex Structure of Amplitudes}
Why is the complex structure of the partial wave amplitudes $f = \frac{1}{-\tilde{k} - ik}$ so important? Let us look at where the poles in the complex plane occur here - it will be instructive, even though the energies are of course real. Our equation for the pole is:
\begin{equation}
    \tilde{k} - ik = 0
\end{equation}
so we say the following:
\begin{equation}
    k = i\tilde{k}
\end{equation}
So:
\begin{equation}
    \frac{2mE}{\hbar^2} = k^2 = -\tilde{k}^2 = -\frac{2m\abs{E}}{\hbar^2}
\end{equation}
the pole in the complex plane sits at the place where the energy $E$ (of the incoming particle) in the complex plane corresponds to the bound state energy. So, if (big if) we can compute the amplitude precisely, we can figure out these poles. This can be done exactly for the Coulomb potential, and the poles will be precisely the hydrogen atom bound states.

\subsection{Further Comments on Resonance, Breit-Wigner Formula} %Breit-Wigner Formula
We are now done with the course material proper. However there are a few important points left to be said.

We have resonance when $\delta = \pi/2$. What happens when $\delta = n\pi$? In this case $\sigma \to 0$.

If we have one wall, then as we increase energy then the tunneling probability increases. What happens with two? Of course for most energies this further decreases the tunneling probability. However, at energy, the wave goes right through the potentials,with probability one! This is the so-called Ramsaner-Townsend effect. This was observed in 1923.

We derived $f$ for a very specific case. In general:
\begin{equation}
    f = -\frac{1}{k}\frac{\frac{\Gamma}{2}}{\left((E - E_R) + i\frac{\Gamma}{2}\right)}
\end{equation}
and
\begin{equation}\label{eq-BWformula}
    \sigma = \frac{4\pi}{k^2}\frac{\frac{\Gamma^2}{4}}{(E - E_R)^2 + \frac{\Gamma^2}{4}}
\end{equation}
the point is that when we are close to the resonance, we have $\frac{4\pi}{k^2}$. $\Gamma$ is the so-called ``width'' of the resonance. At the resonance, we saturate the unitarity limit. Mathematically:
\begin{equation}
    E = E_R - i\frac{\Gamma}{2}
\end{equation}
what is the meaning of this? The wavefunction is as:
\begin{equation}
    \psi \sim e^{-i\frac{Et}{\hbar}} \to e^{-i\frac{E_R t}{\hbar}}e^{-\frac{\Gamma t}{2}}
\end{equation}
so:
\begin{equation}
    \abs{\psi}^2 \sim e^{-\frac{\Gamma t}{\hbar}}
\end{equation}
we have a so-called metastable state. These are the states we discussed when we discussed transition. In pure QM, all states are absolutely stable - we have to go to QFT to understand instability (and we did this, albeit in a cheaty way). In this formula, we have a lifetime $\tau = \frac{\hbar}{\Gamma}$. This is the way our metastable states decay, and it can be formulated as the $E = E_R - i\frac{\Gamma}{2}$ condition (even though of course the energies are actually real).

The formula Eq. \eqref{eq-BWformula} is very important for many fields of physics! Remember it.