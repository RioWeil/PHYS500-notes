\section{Corrections to the Hydrogen Atom Continued}
\subsection{Review of Last Lecture}
We started to discuss a real hydrogen atom which exists in nature. We did estimates for the so-called ``fine structure'' of the hydrogen atom. We call:
\begin{equation}
    \alpha = \frac{e^2}{\hbar c}
\end{equation}
the fine structure coupling constant. The relativistic and spin-orbit corrections were of order $\alpha^2$. The hyperfine structure (which we will discuss in a few lectures) was of order $\alpha^210^{-3}$. There are also a few other effects; for example the Zeeman splitting from the Earth's magnetic field, and the Lamb shift which can be calculated from QFT (virtual pair production\ldots) which is given by:
\begin{equation}
    \Delta E = \frac{\alpha^3}{6\pi}\si{Ry}\ln\left(\frac{1}{\pi\alpha}\right)
\end{equation}
Note that this has been measured, and this was computed by Feynman to high precision; leading to the birth of QFT! 

\subsection{Relativistic Correction - Computation}
We calculated the relativistic correction last class:
\begin{equation}
    \Delta E_{rel}^{(1)} = -\frac{1}{2mc^2}\left[E_n^2 + 2E_n\avg{\frac{e^2}{r}} + \avg{\frac{e^4}{r^2}}\right]
\end{equation}
where $E_n = \si{Ry}\left(-\frac{1}{2n^2}\right)$. The logic behind this was $\ket{n, l, m}$ are eigenstates of the unperturbed Hamiltonian:
\begin{equation}
    H^{0} = \frac{p^2}{2m} + V(r)
\end{equation}
We could then express $p^4$ (the relativistic correction) in terms of $H$ and hence obtain an expression in terms of eigenenergies. The remaining step is to compute the expectation values.

\subsubsection{The Feynman-Hellman Theorem}
Let:
\begin{equation}
    E_n(\lambda) = \bra{n}H(\lambda)\ket{n}.
\end{equation}
Then:
\begin{equation}
    \dpd{E_n}{\lambda} = \bra{n}\dpd{H}{n}\ket{n}
\end{equation}

\textit{Proof.} We compute using the product rule:
\begin{equation}\label{eq-productruleFeynman}
    \begin{split}
        \dpd{E_n}{\lambda} = \left(\dpd{}{\lambda}\bra{n}\right)H(\lambda)\ket{n(\lambda)} + \bra{n}\dpd{H}{n}\ket{n} + \bra{n}H\ket{\dpd{n}{\lambda}}
    \end{split}
\end{equation}
Now, since $\braket{n}{n} = 1$, it follows that:
\begin{equation}
    \dpd{}{\lambda}\braket{n}{n} = 0.
\end{equation}
And therefore:
\begin{equation}
    \dpd{}{\lambda}\braket{n}{n} = \left(\dpd{}{\lambda}\bra{n}\right)\ket{n} + \braket{n}{\dpd{n}{\lambda}} = 0
\end{equation}
and so the first/last terms in Eq. \eqref{eq-productruleFeynman} vanish, and we are left with the result. \qed

\subsubsection{Applying the Feynman-Hellman Theorem}
We have the known standard integral:
\begin{equation}
    \int_{-\infty}^\infty e^{-ax^2}dx = \sqrt{\frac{\pi}{a}}
\end{equation}
which can be computed by the trick of going into polar coordinates. We can then obtain the solution for whole families of integrals, e.g.:
\begin{equation}
    \int_{-\infty}^\infty x^2e^{-ax^2}dx = \dpd{}{a}\sqrt{\frac{\pi}{a}}
\end{equation}

\subsubsection{Applying the Feynman-Hellman Theorem for Relativistic Corrections}
We have:
\begin{equation}
    H^{(0)} = -\frac{\hbar^2}{2mr^2}\dpd{}{r}\left(r^2\dpd{}{r}\right) + \frac{\hbar^2l(l+1)}{2mr^2} - \frac{e^2}{r}
\end{equation}
Now we observe:
\begin{equation}
    \dpd{E_n}{(e^2)} = -\frac{2me^2}{2\hbar^2}\frac{1}{n^2} = \avg{-\frac{1}{r}}
\end{equation}
How did we know to pick this? We have to be a bit smart and choose to differentiate with respect to the variable that gives us the result we want. Next, we will want to differente w.r.t $l$. Recall however that $n$ depends on $l$:
\begin{equation}
    n = n_r + l + 1
\end{equation}
where $n_r$ is the radial quantum number and $l$ is the angular (and $n$ is the composite, principle quantum number). When we differentiate w.r.t. $l$, we get:
\begin{equation}
    \dpd{H^{(0)}}{l} = \frac{(2l+1)\hbar^2}{2mr^2}
\end{equation}
\begin{equation}
    \dpd{E_n}{l} = \frac{me^4}{\hbar^2(n_r + l + 1)^3}
\end{equation}
Since the expectation value of the first term must be equal to the second by the theorem, we can solve for what $\avg{\frac{1}{r^2}}$ should be. If we put everything in, we get:
\begin{equation}
    \Delta E_{rel}^{(1)} = -\frac{1}{2mc^2}\left[E_n^2 + 2E_n\avg{\frac{e^2}{r}} + \avg{\frac{e^4}{r^2}}\right] = \left(\frac{me^4}{\hbar^2}\right)\frac{\alpha^2}{(2n^2)^2}\left(-\frac{1}{2}\right)\left(\frac{4n}{l+\frac{1}{2}} - 3\right)
\end{equation}

So we get the $\propto\alpha^2$ from our order of magnitude calculation! This is not at all surprising, because all the terms we add together are of the same order, and:
\begin{equation}
    \frac{1}{mc^2}\left(\frac{me^4}{\hbar^2}\right)^2 = \left(\frac{me^4}{\hbar^2}\right)\left(\frac{me^4}{\hbar^2mc^2}\right) = \alpha^2
\end{equation}

Some important notes: we see a dependence on $l$ in our relativistic correction, and of course it should appear in all of our computations. 2s and 2p orbitals would have different corrections. Second remark; why can we use PT without degeneracy when we have huge degeneracy of $g = 2n^2$? Because our perturbing Hamiltonian commutes with our classification scheme.

\subsection{Spin-Orbit Correction - Calculation}
We start with the Hamiltonian:
\begin{equation}
    H_{SO} = -\gv{mu} \cdot \v{B}
\end{equation}
where $\v{B}$ is a classical (dipole - from multipole expansion) field:
\begin{equation}
    \v{B} = \frac{\abs{e}}{c}\frac{\v{v} \times \v{r}}{r^3} = \frac{\abs{e}}{\hbar c}\frac{\v{L}}{r^3}
\end{equation}
and we have:
\begin{equation}
    \mu = -\frac{\abs{e}\hbar}{2mc}(2\v{S})
\end{equation}
therefore:
\begin{equation}
    H_{SO} = \frac{e^2\hbar^2}{2m^2c^2}\frac{\v{S} \cdot \v{L}}{r^3}.
\end{equation}
We now define $\v{J} = \v{L} + \v{S}$ as usual, and observing that:
\begin{equation}
    [H_{SO}, \v{J}_i] = 0
\end{equation}
we conclude that we can use the $\ket{n, l, s=1/2, j, j_z}$ basis/classification in our perturbation theory, and we can use the non-degenerate PT technique as $\v{J}$ commutes with the Hamiltonian. We are interested in:
\begin{equation}
    \bra{n,l,s=1/2, j, j_z}\v{L} \cdot \v{S}\ket{n,l,s=1/2, j, j_z}
\end{equation}
But since:
\begin{equation}
    \v{L}\cdot \v{S} = \frac{1}{2}\left(\v{J}^2 - \v{S}^2 - \v{L}^2\right)
\end{equation}
we then have:
\begin{equation}
    \bra{n,l,s=1/2, j, j_z}\v{L} \cdot \v{S}\ket{n,l,s=1/2, j, j_z} = \frac{1}{2}\left(j(j+1) - \frac{3}{4} - l(l+1)\right).
\end{equation}

We still need to compute $\avg{\frac{1}{r^3}}$. We can use Kramer's Relation (whose proof comes down to integration by parts):
\begin{equation}
    \avg{\frac{1}{r^3}} = \frac{1}{a_B^3}\frac{1}{n^3(l + \frac{1}{2})l(l+1)}
\end{equation}
From this we can obtain the full solution:

\begin{equation}
    \avg{H^{(1)}} = \frac{e^2\hbar^2}{2m^2c^2}\frac{1}{a_B^3}\frac{j(j+1) - \frac{3}{4} - l(l+1)}{2n^3(l + \frac{1}{2})l(l+1)}
\end{equation}
But let's look at the constant prefactor here:
\begin{equation}
    \frac{e^2\hbar^2}{2m^2c^2}\left(\frac{me^2}{\hbar^2}\right)^3 = \frac{me^4}{2\hbar^2}\left(\frac{e^2}{\hbar c}\right)^2 = \si{Ry}\alpha^2
\end{equation}
so again we see that our order of magnitude prediction of $\propto \alpha^2$ checks out!

\subsection{Combining the Corrections}
In sum, the fine structure corrections to the hydrogen atom are given by:
\begin{equation}\label{eq-finestructurecorrection}
    \Delta E_{SO}^{(1)} + \Delta E_{rel}^{(1)} = -\frac{me^4}{\hbar^2}\frac{\alpha^2}{2n}\left[\frac{n}{j+\frac{1}{2}} - \frac{3}{4}\right]
\end{equation}
If we plot the energy levels with the fine structure corrections, we obtain Fig. \ref{fig-finestructuresplit}.

\begin{figure}[htbp]
    \centering
    Figure of energy splittings
    \caption{<caption>}
    \label{fig-finestructuresplit}
\end{figure}

We notice something very intriguing about the energy correction in Eq. \eqref{eq-finestructurecorrection}; the net result does \emph{not} have any dependence on $l$ (only on $j$). It is extremely nontrivial. For any potential in the world, we would have an $l$ dependence; but for this \emph{very} special (unique! - for further reading see the Runge Lentz vector and its connection to $1/r^2$ potentials) case, we have this $l$-indepenent structure. 